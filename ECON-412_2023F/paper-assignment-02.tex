 \noindent
\begin{minipage}[t]{.6\textwidth}
\raggedright
	\Large ECON 412 Paper Assignment 2 \\
 	\large [Your team members' names] \\
 	\today \\[1.5em] % today's date, then gives a little space below
\end{minipage}% <-- Don't forget this one
%
\begin{comment}
\hfill
\begin{minipage}[t]{.4\textwidth}
\vspace{-1em}
\raggedright 
\begin{figure}[H]
	\includegraphics[width=.3\linewidth]{figures/did-best-no-regrets.png} % insert a team logo if you have time and energy, why not
\end{figure}
\end{minipage}
 
 \vspace{1em} 
 I thank my mother and father and all my friends for continued collaboration and the plentiful tips from section.\footnote{Picture source: \url{https://lparchive.org/Pokemon-Yellow/Update}}
 
\end{comment}

\section{Topic 1}


For more see \url{https://www.overleaf.com/learn/latex/Bold,_italics_and_underlining}


These parts are \textbf{emboldened} 
or \underline{underlined} 
and also \textbf{\textit{bold and italicized}}.

You can also italicize with \emph{this command}. The \verb+emph+ and \verb+textit+ behave slightly differently.
\subsection{Why is this interesting}

\subsection{Possible Questions of Interest}

\subsection{Possible Data Sources}

\subsection{Common symbols}

If you want to insert an \& or \$ , you need a back slash \ before you do so, since Latex reserves these for alignment commands and math environment commands respectively.



\subsection{IMPORTANT NOTE}
You should never manually type numbers of equations, tables, figures, page numbers, citations, etc. Let Latex handle this referencing and labeling for you.\\

To add a list, use

\begin{enumerate}
    \item This is an item
\end{enumerate}

or 

\begin{itemize}
    \item This is an item
\end{itemize}

For a citation, you typically either want \citet{taylorBuffaloHuntInternational2011} for in-text citations, or at the end of a sentence \citep{taylorBuffaloHuntInternational2011}.