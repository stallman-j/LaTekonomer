 \noindent
\begin{minipage}[t]{.8\textwidth}
\raggedright
	\Large ECON 412 Paper Assignment 3, due October 11th \\
 	\large [Your team members' names] \\
 	\today \\[1.5em] % today's date, then gives a little space below
\end{minipage}%


\section{Title}
With your team and in consultation with Professor Kortum and/or Jillian, select which of your proposed paper topics you’ll pursue and provide your working title.

\section{Research Question}

What \textit{primary} research question will you seek to answer? Review abstracts of economics papers you’ve seen to get an idea of how research questions are framed, e.g.  what is the effect of economic growth on environmental quality?

\section{Conceptual Framework (Pre-Analysis Plan)}

If you’re making a causal claim, what’s your identification strategy (difference in difference, before-after analysis, regression discontinuity, instrumental variables)? If you’re not making a causal claim, what is your strategy for making a persuasive argument? If you are taking a more theory/model-based approach (such in the first two psets):  what is the model, what can it help you explain, and how might you estimate or calibrate its parameters?

What is the \textbf{primary} regression you plan to run? This should follow directly from your research question. Write out an equation like in Lecture 6a. What’s your unit of analysis? (e.g. countries, firms, individuals?) What’s your extent of time? (e.g. cross section (a single moment in time), annual panel, monthly panel) 

Define your outcome variable(s) and regressor(s). For each regressor, say whether they’re acting to control for potential omitted variable bias or helping increase precision (see slides 6a). Will you need to take any transformations of the variables (e.g. logs)?


\section{Data}
What data do you need to run the regression you described in Question 3? By this time, you should have acquired or had approved a plan to acquire the data you need for \textit{each} of the regressors and your outcome variable(s).

Describe one or two figures that will help illustrate your argument. You can do this in words, with a sketch, or just input the plots if you’ve reached this point in your analysis.

Describe two tables that you plan to show in your in-class presentation. Input the tables if you’ve reached this point in your analysis; if not, adapt the tables provided in Section~\ref{sec:latex_code} with the column titles and rownames for the tables you plan to make, leaving the cells to have bogus values.\footnote{Please note if your table has true or nonsense values.}


\section{Potentially Helpful Latex Code}\label{sec:latex_code}
You can find some Latex code in the Overleaf Templates \href{https://www.overleaf.com/read/mpdhvnnjzsxq}{(click here)}, \verb+paper-assignment-03.tex+, under the \verb+\begin{comment}+ below. Note that each table provided in the \verb+\input{}+ command is the output of a Stata or R command.


\subsection{Equation}
Equation~\ref{eq:main_spec} describes our primary specification.

\begin{equation}\label{eq:main_spec}
    Y_{it} = \beta_0 + \beta_1 X_{it} + \beta_2' Z_{it} + \gamma_t + \alpha_i + \varepsilon_{it}
\end{equation}

\subsection{Figures}

Figure~\ref{fig:sub1} suggests someone grew up with a Gameboy, while Figure~\ref{fig:sub2} is redundant. Figure~\ref{fig:fig3} is overkill.

% if you are getting angry red dots below, insert 
%\usepackage{subcaption} and \usepackage{caption}
% and comment out the \usepackage{subfigure} with a %
% in your preamble

\begin{figure}[H]
\centering
\begin{subfigure}[h]{0.5\textwidth}
  \centering
  \includegraphics[width=.9\textwidth]{figures/did-best-no-regrets.png}
  \caption{Subfig 1}
  \label{fig:sub1}
\end{subfigure}%
\begin{subfigure}[h]{0.5\textwidth}
  \centering
  \includegraphics[width=.9\textwidth]{figures/did-best-no-regrets.png}
  \caption{Subfig 2}
  \label{fig:sub2}
\end{subfigure}
\caption{Predicted Capacity, MW and log(MW)}
\label{fig:fig_01}
\end{figure}

\begin{figure}[H]
	\centering % centers
	\includegraphics[width=.6\linewidth]{figures/did-best-no-regrets.png}
	\caption{Figure Caption} % note caption is required in order to get the cross referencing
	\label{fig:fig3}
\end{figure}


For demos on using stata to excel and stata to Latex, see \href{https://github.com/worldbank/stata-tables}{The World Bank DIME's stata-table github.} For R, see \href{https://www.jakeruss.com/cheatsheets/stargazer/#the-default-summary-statistics-table}{the stargazer package here.}


\subsection{Tables}

Table~\ref{tab:ghg_sumstats} shows some summary statistics for the problem set data:

\begin{table}[H]
\centering
\caption{Summary Statistics}
\footnotesize
{
\def\sym#1{\ifmmode^{#1}\else\(^{#1}\)\fi}
\begin{threeparttable}


% Table created by stargazer v.5.2.3 by Marek Hlavac, Social Policy Institute. E-mail: marek.hlavac at gmail.com
% Date and time: Fri, Oct 06, 2023 - 9:52:24 AM
\begin{tabular}{@{\extracolsep{5pt}}lcccccc} 
\\[-1.8ex]\hline 
\hline \\[-1.8ex] 
Statistic & \multicolumn{1}{c}{N} & \multicolumn{1}{c}{Min} & \multicolumn{1}{c}{Mean} & \multicolumn{1}{c}{Median} & \multicolumn{1}{c}{Max} & \multicolumn{1}{c}{St. Dev.} \\ 
\hline \\[-1.8ex] 
Territorial Emissions (MtC/yr) & 10,235 & 0.00 & 34.03 & 2.02 & 2,931.49 & 155.17 \\ 
Consumption Emissions (MtC/yr) & 3,587 & $-$8.85 & 62.92 & 10.13 & 2,718.11 & 216.29 \\ 
Emissions Transfers (MtC/yr) & 3,587 & $-$180.01 & $-$0.01 & $-$0.48 & 416.62 & 28.54 \\ 
Population (000s) & 10,349 & 4.26 & 30,875.98 & 6,167.44 & 1,419,730.00 & 114,930.50 \\ 
GDP (chained PPPs,mil.2017USD) & 10,349 & 20.36 & 306,313.40 & 30,597.50 & 20,860,506.00 & 1,217,080.00 \\ 
GDP per capita & 10,349 & 240.70 & 13,039.99 & 6,424.18 & 283,927.30 & 18,951.75 \\ 
Energy Emissions (MtCO2/yr) & 3,739 & 0.08 & 133.79 & 33.36 & 2,469.49 & 233.60 \\ 
Oil Consumption Emissions (MtCO2/yr) & 5,698 & 0.04 & 39.86 & 10.16 & 689.99 & 83.11 \\ 
\hline \\[-1.8ex] 
\end{tabular} 


  \begin{tablenotes}
    %\item[$*$] $p<0.1$, \sym{**} $p<0.05$, \sym{***} $p<0.01$
    %\item[\dag] These ... \smallskip
    \item \emph{Notes:} Territorial, consumption, and emissions transfers are obtained from an update to \citet{petersSynthesisCarbonInternational2012}, and are given in million tonnes of carbon per year. One million tonnes of carbon equal 3.664 million tonnes of CO2 equivalent. Population in thousands are obtained from \citet{undesaWorldPopulationProspects2022}. Gross domestic product (GDP) is expenditure-side real chained GDP at purchasing power parity (PPP) from \citet{feenstraNextGenerationPenn2015}. Energy and oil consumption emissions in million tonnes of CO2 equivalent from \citet{ieaGreenhouseGasEmissions2023}. Emissions data varies by country-year within 1950 to 2019, the range in which population and GDP data are available.
  \end{tablenotes}

\end{threeparttable}
}
\label{tab:ghg_sumstats}
\end{table}

\begin{comment}

    This gives an example of inputting a table from stata output:

Table~\ref{tab:econ412_tab01} presents the results from Equation~\ref{eq:main_spec}.

\begin{table}[htbp]\centering
\def\sym#1{\ifmmode^{#1}\else\(^{#1}\)\fi}
\caption{(Your Title Here) \label{tab:econ412_tab01}}
\begin{tabular}{l*{1}{D{.}{.}{-1}}}
\toprule
                    &\multicolumn{1}{c}{(1)}         \\
\midrule
primary var                  &                     \\
main control            &     886.304  \\
                    &   (487.462)         \\
other control                  &                     \\
Constant            &    5090.048\\
                    &   (277.042)         \\
\midrule
Observations        &         722         \\
\bottomrule
\multicolumn{2}{l}{\footnotesize Standard errors in parentheses}\\
\multicolumn{2}{l}{\footnotesize Other notes.}\\
\multicolumn{2}{l}{\footnotesize Data sources: ().}\\
\end{tabular}
\end{table}


Table~\ref{tab:sumstats} shows the summary statistics for ...


\begin{table}[htbp]\centering \caption{Summary Statistics  \label{tab:sumstats}}
\begin{tabular}{l c c c c c}\hline
\multicolumn{1}{c}{\textbf{Variable}} & \textbf{Mean}
 & \textbf{Std. Dev.}& \textbf{Min.} &  \textbf{Max.} & \textbf{N}\\ \hline
1-5 Mortality Index & 0.947 & 2.305 & 0 & 23.5 & 523\\
Trade Shock & 0.319 & 0.645 & 0 & 2.529 & 523\\
Grain price & 0.68 & 0.213 & 0.288 & 1.243 & 364\\
Position & 0.02 & 0.02 & 0 & 0.179 & 523\\
Non-Han & 0.037 & 0.029 & 0 & 0.185 & 523\\
Confucian & 4.595 & 6.612 & 0 & 21 & 523\\
Buddhist & 2.778 & 5.367 & 0 & 18 & 523\\
\hline\end{tabular}
\end{table}


\end{comment}

