 \noindent
\begin{minipage}[t]{.8\textwidth}
\raggedright
	\Large ECON 412 Paper Assignment 4: Presentations \\
 	%\large [Your team members' names] \\
 	\today \\[1.5em] % today's date, then gives a little space below
\end{minipage}%

\noindent

\textbf{\Large{Presentation Guidance}}

We'll have in-class presentations on Monday, October 30th and Wednesday, November 1st during the first half of class each day. You'll have six minutes with your group to present, followed by a minute for comments and questions. If you're not presenting, you're encouraged to ask questions during this Q\&A. Each member of your group should speak. Part of your presentation grade will also consist of providing feedback on the presentations of your peers through a Canvas discussion. Your presentation can employ any format (e.g. Latex Beamer,\footnote{See Section~\ref{sec:beamer} for resources for Beamer} Microsoft Powerpoint, or Mac's Keynote).

You should both 1) send your presentation slides to Professor Kortum (CC Jillian) and 2) upload your presentation slides in the ``paper assignment 4'' post on Canvas in PDF form by \textbf{midnight the night before your presentation}.


\section{Presentation Components}
Economics presentations tend to be quite formulaic. You start with a hook or motivation that answers the question of why the audience should care about this topic, describe the research question you're trying to answer, tell us how you'll go about answering your research question (data and methods), provide results, and conclude with a summary and/or future research directions.

Your presentation should therefore include the following components:\footnote{Section~\ref{sec:runofshow} provides an example sequence of slides you might want to use.}


\begin{enumerate}
    \item A title and names of your team members
    \item Motivation of why you're exploring this topic
    \item Statement of your research question 
    \item Description of your methodology
    \item Description of your data sources
    \item At least one table or figure that helps to make your point. This is ideally a figure or table generated from your own data and would ideally show some portion of your working results. If your data require extensive cleaning or your code/model is complex, this could be from another source, or you may just show us summary statistics. If you do use an external figure, please explain why you're not showing a table or figure of your own creation.
    \item A description of what you plan to accomplish between your presentation and the final paper
\end{enumerate}

\section{Constructive Criticism Requirement}

\begin{enumerate}
    \item \textbf{Before} the day of your presentation, your team should reply to the discussion called ``Project Feedback'' in the Canvas Discussions tab in the following way:
    \begin{itemize}
        \item Title: Title of your project
        \item Abstract: Include an abstract of no more than 250 words.
    \end{itemize}
    \item By \textbf{11:59pm on November 8th}, each member of your team should provide constructive criticism on \textbf{minimum two projects} which are not your own, in the form of a reply to the post the other team made about their project. You shouldn't comment on the same project as what your team members did (e.g. within a team of three, your team cumulatively would provide feedback on six different projects). There's no minimum length requirement for your posts, but you should include at least one tangible recommendation for improvement on each post.
\end{enumerate}

\noindent

\textbf{Additional Guidance:} 

An abstract typically includes a hook for why the reader should care, a statement of the research question, a description of the innovation relative to prior literature, a description of the methodology (and perhaps the data), and the headline result. At this point in your project you might write your expected result rather than an actual result. If a member of your team is planning on continuing with this project beyond the semester, you may want to note this in your post below your abstract.

You're encouraged to discuss with your team members and make your post reflect your entire team's feedback, just note the people you discussed with. Please try to ensure that no one project gets disproportionate attention.

You may for diplomatic purposes wish to sweeten your constructive criticism with a note about a positive aspect of the project or presentation. You can provide critique on the presentation per se (being sensitive about singling out individuals for critique) or about the broader project. 

You're encouraged but not required to discuss the projects on which you're providing feedback with members of other teams outside of the presentations so that you have more information with which to flesh out your feedback. 

If you see it noted that the project will be continued past the semester, your constructive criticism will be especially valuable at this point of their project, so you may want to focus more on the weaknesses than the strengths. Still be diplomatic, of course.

\section{Resources}

\subsection{Possible Run of Show}\label{sec:runofshow}
Your team is not required to abide by the following format, but the following sequence provides a possible run of show for your presentation.

\begin{itemize}
    \item[1)] \textbf{Title slide:} include at least your title, team member names, and date ($\approx$ 30 seconds)
    \item[2)] \textbf{Motivation:} this is your hook that draws the audience in. For example, this could be a figure that summarizes your issue,\footnote{One of the figures you describe in paper assignment 3, for instance.} a graph from your analysis that makes your point clearly, a snapshot from a news article, or a picture that creates an emotional connection to the issue. ($\approx$ 1 minute)
    \item[3)] \textbf{Overview:} A slide that summarizes your project at a glance. What's the research question; what's your strategy (e.g. is it a model, are you doing a difference-in-difference regression?); what innovations are you making on prior literature; what's the context and data. Often helpful to include a key equation here, e.g. if you have an instrumental variables approach, what's your primary specification? ($\approx$ 1.5 minutes)
    \item[4)] \textbf{Results:} If you can, show a table and/or figure that demonstrates some part of your results. You likely don't have your full results at this point, but try to have at least some portion of your data cleaned enough that you have a figure or table to show us.\footnote{The tables and figures you described in paper assignment 3 would be great candidates for this slide or slides.} ($\approx$ 1.5 minutes)
    \item[5)] \textbf{Future work: }Tell us what your goals are for the time between this presentation and the final product. You might also throw in a sentence or two about future research directions that you recognize you won't likely get to, but would be viable in the future or building off your project. ($\approx$ 1 minute)
    \item[6)] \textbf{Conclusion:} You might use a slide to summarize, but you could also just discuss this in a few sentences along with your future work since your overview slide already gave your summary. ($\approx$ 30 seconds)
\end{itemize}


\subsection{General Presentation Tips}
\begin{itemize}
    \item Practice beforehand! Particularly with a group, walking through your presentation beforehand will allow you to be fluid when you're up in front of the class and will allow you to control your time to hit this quite stringent limit. Ideally you would have at least one run-through to an audience, preferably including at least one run-through involving a non-economics audience member who will tell you when your jargon is too technical or you're just not making sense. If you do get that feedback, ask yourself if the jargon was necessary (it is sometimes!) and try to make it more accessible if not. You're welcome to do a test run in office hours or section with Jillian.
    \item Expect to require on average 1-1.5 minutes per slide. Don't write the total slide count on your presentation (otherwise it feels like a countdown for the audience). For a six-minute presentation, four to eight slides should suffice. 
    \item Look at the audience while you're presenting, and try to face the audience unless you're pointing out something in particular on a slide or making a specific point where you need to turn your body. You should be well practiced enough that you don't need to read off your slides. 
    \item If you get nervous in front of a group, extra practice ahead of time will help ease those nerves. If making eye contact is intimidating, choose someone who you know who will give you supportive expressions while you're talking, or look over someone's shoulder if making eye contact will throw you off. It'll look to the audience like you're making eye contact.
    \item The audience puts priority on what's said \textit{first}, and what's said most \textit{recently}, so start and end with your most convincing statements or what you want us to leave the room with.
    \item You'll gain credibility with your audience if you avoid trying to oversell what you can answer. Economists are notoriously skeptical, tough crowds, so if you confidently try to tell us that you're answering a broad and complex question, it might undermine your results because you'll have lost trust with us. In short, be honest about what you are and aren't able to do. You might even want to tell us what the weaknesses are in your methodology. This is a very effective persuasion tactic: if you admit your own strategy's shortcomings, we're more likely to believe your results because we don't suspect you were trying to hoodwink us at the outset.
    \item If you're an audience member and wondering whether you should ask a question or make a comment during the Q\&A, the answer is yes, please do. If you have a question, chances are that someone else in the class audience has a similar one and you'll generate positive externalities. Stupid questions surely exist, but this class has yet to ask one. 
\end{itemize}

\subsection{Beamer}\label{sec:beamer}
If you wish to use Beamer, a video tutorial of using Beamer in Latex can be found in the \verb+resources/videos+ folder at this link.

\begin{itemize}
    \item \href{https://u.pcloud.link/publink/show?code=kZSldiVZk9tFHPNfX4jmPgM6LCwLipqfczey}{https://u.pcloud.link/publink/show?code=kZSldiVZk9tFHPNfX4jmPgM6LCwLipqfczey}
    \item \textbf{Password:} \verb+2023_Fall_ECON-412+
\end{itemize}

and a template can be found in \texttt{templates\_folder/presentations\_templates} in the Overleaf Templates project at the following:

\begin{itemize}
    \item \href{https://www.overleaf.com/read/mpdhvnnjzsxq}{https://www.overleaf.com/read/mpdhvnnjzsxq}
\end{itemize}