 \noindent
\begin{minipage}[t]{.8\textwidth}
\raggedright
	\Large ECON 412 Paper Assignment 5: Rough Draft \\
 	%\large [Your team members' names] \\
 	Due Wednesday November 15th, 11:59pm \\[1.5em] % today's date, then gives a little space below
\end{minipage}%

\noindent

\textbf{\Large{Paper Draft Guidance}}


\section{Components}
Your rough draft should include the following components. These should build naturally on your previous paper assignments and presentations.

\begin{enumerate}
    \item A title and names of your team members
    \item Motivation for exploring this topic
    \item Statement of your research question
    \item Description of your original contribution to the question
    \item Description of your methodology
    \item At least one equation that helps illustrate your strategy
    \item Description of your data sources
    \item At least one original table and one original figure that helps to make your point. The final draft will require $\geq$ two original tables and $\geq $ two separate plots (which may be placed side-by-side into a single figure).
    \item Interpretation of your current results
    \item A brief statement of what you plan to accomplish between this draft and the final paper
\end{enumerate}

\section{Formatting}

\textbf{Final} papers will have the following formatting requirements
\begin{itemize}
    \item no longer than six pages
    \item minimum single-spaced
    \item minimum font size 12
    \item minimum one-inch margins
\end{itemize}

For \textbf{rough drafts}, you won't lose points and can request feedback on up to \textbf{eight} pages of text, along with a reasonable number (single digits) of appendix figures or tables if you'd like input on what results are compelling to display. You're encouraged to stick to the final draft's six pages, but you'll likely find that it's easier to go over than under. This page count includes your two tables and two figures. References separate.

The default settings of the previously provided Overleaf Template would suffice. You're highly encouraged but not required to submit your rough draft as a PDF generated from Latex. Please discuss with Professor Kortum or Jillian if your group strongly prefers to submit both your rough and final drafts in a PDF obtained from Microsoft Word, as the default expectation will be Latex. 

This won't be explicitly policed, but if your final draft is greater than 3,000 words, you're probably writing too much. Eight pages of text would correspond to up to about 4,000 words, although likely much less when figures and tables are included. 


\section{Additional Suggestions}

You're not required to abide by the following guidance, but here are some suggestions for effective short-paper writing. 
\begin{enumerate}
    \item Don't wait to tell the reader what your contribution is by providing excessive background up front. A good rule of thumb is: paragraph 1 tells us broadly why this issue is interesting. Paragraph 2 tells us why the narrower issue is relevant. Paragraph 3 then describes what you're doing about it.
    \item The length and requirements of this paper are roughly similar to an academically oriented blog post. \href{https://cepr.org/voxeu}{VoxEU} or \href{https://voxdev.org/}{VoxDev} provide good examples of this style of writing
    \item In general, it's often useful to pitch your jargon and explanation to you or your classmates of about a year prior. A well-educated non-economist should be able to follow along, but might have to gloss over a few words (e.g. assume the reader is versed in basic statistics and understands identification). 
    \item It often helps to couch the magnitudes of your results in terms of means and standard deviations. For instance, starting at the mean value of your outcome variable, consider how much the predicted outcome would increase if you were to increase your dependent variable of interest by one standard deviation, holding all other independent variables constant. Is this predicted increase greater or less than a standard deviation of the outcome variable? This is one common way to measure ``economic'' or ``real-world'' significance. You might also speculate on whether these marginal effects would differ at different starting levels (rather than the mean)
    \item You may also want to interpret your results in terms of stylized choices of observations. This takes some effort but greatly helps the reader understand your results. For instance, from Homework Assignment 3 a strong intuitive interpretation might have been to pick a country-year (call this country $A$ in year $t$) that had GDP per capita at roughly the mean level. Then examine one standard deviation of GDP per capita, and find 1) if the country has seen GDP growth, what year the country achieved growth such that its GDP per capita was at one standard deviation of the mean and 2) another country $B$ that has GDP approximately one standard deviation above the mean in year $t$. You could then compare how much of a predicted increase would occur in greenhouse gases per capita in moving country $A$ from year $t$ to the year in which it had this increased GDP per capita; or the predicted difference in greenhouse gases per capita between countries $A$ and $B$.
    \item Figures and tables should be self-standing. Your notes and captions or axis labels should include the relevant units, definitions, and data sources so that you could show \textit{just} the figure or table to a reader who hadn't read your text, and they would be able to interpret it.
\end{enumerate}


\section{Other Resources}

Latex templates can be found in the usual Overleaf Templates project at the following:

\begin{itemize}
    \item \href{https://www.overleaf.com/read/mpdhvnnjzsxq}{https://www.overleaf.com/read/mpdhvnnjzsxq}
\end{itemize}