\section{General Writing Template}\label{sec:general-writing}


For more see \url{https://www.overleaf.com/learn/latex/Bold,_italics_and_underlining}


These parts are \textbf{emboldened} 
or \underline{underlined} 
and also \textbf{\textit{bold and italicized}}.

You can also italicize with \emph{this command}. The \verb+emph+ and \verb+textit+ behave slightly differently.

\subsection{Common symbols}

If you want to insert an \& or \$ , you need a back slash \ before you do so, since Latex reserves these for alignment commands and math environment commands respectively.



\subsection{IMPORTANT NOTE}
You should never manually type numbers of equations, tables, figures, page numbers, citations, etc. Let Latex handle this referencing and labeling for you.\\

To add a list, use

\begin{enumerate}
    \item This is an item
\end{enumerate}

or 

\begin{itemize}
    \item This is an item
\end{itemize}

For a citation, you typically either want \citet{taylorBuffaloHuntInternational2011} for in-text citations, or at the end of a sentence \citep{taylorBuffaloHuntInternational2011}.


\section{Math}
If you want all the great math symbols, see \url{https://oeis.org/wiki/List_of_LaTeX_mathematical_symbols}. If you can't find something in there, check out \url{http://tug.ctan.org/info/symbols/comprehensive/symbols-a4.pdf}.

For a pretty empty set, put \verb+\usepackage{amssymb}+ in your preamble, and use $\varnothing$ rather than $\emptyset$ or $\{\}$.


For script characters, use \verb+\mathcal{P}+ for, for instance, $\mathcal{F}$. \\

Note that there are two epsilons, \verb+epsilon+ ($\epsilon$) and \verb+varepsilon+ ($\varepsilon$)
