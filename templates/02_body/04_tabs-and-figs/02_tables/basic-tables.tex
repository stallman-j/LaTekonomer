\section{Tables}\label{tables_basics}

See \url{https://www.overleaf.com/learn/latex/tables} for the basics.


For some best practices on sending tables to Latex, see the \href{https://blogs.worldbank.org/impactevaluations/nice-and-fast-tables-stata}{World Bank Blog Post here}. In particular, see \href{http://www2.stat.duke.edu/~rcs46/lectures_2015/01-markdown-git/slides/naming-slides/naming-slides.pdf}{these slides} on how to name your files.


See \url{https://tex.stackexchange.com/questions/12672/which-tabular-packages-do-which-tasks-and-which-packages-conflict} for more information on table packages. There are a lot of packages that do table-like things, and oftentimes which ones you use will depend on what the default output (Stata, R, Matlab) uses when it's outputting.\\

See especially that discussion if you find yourself needing tables that span pages, get really long, get really wide, or need captioning.\\

Must-have packages for the academic: \verb+booktabs+, \verb+array+, \verb+tabularx+, \verb+threeparttable+. The first is good for customization and professional-looking tables. The second is general customization and shows up in all sorts of funky places. The third similarly, and the fourth for making tweaks as well. Another common one is \verb+multirow+

If you're sending tables from a statistical package like R or Stata to Latex, DO NOT edit them manually in Latex. Figure out your commands for the statistical package (like outreg for stata) and do all your editing at the script level in the statistical software. Then, you should send the tables to the subfolder "Tables" in whatever directory houses your master latex file, and you can use \verb+\input{Tables/yourtablename.tex}+ to tell Latex to use the table file. 

For compatibility with Stata, see for example the documentation for \verb+outreg2+ at \url{http://repec.org/bocode/o/outreg2.html}. Commands like \verb+estout/esttab+ will get you results quickly, but aren't very customizable. See for example Luke Stein's discussion \href{https://lukestein.github.io/stata-latex-workflows/}{here}. For a higher-cost but beautiful setup, see \verb+tabout+, whose documentation has a bunch of beautiful examples and Stata do files at \href{http://ianwatson.com.au/stata.html}{Ian Watson's Stata page.}

For Stata-to-Latex examples and more information, see section \ref{stata_tables}

\subsection{The very basics}

Here's the simplest example. You need to tell Latex how many columns with \verb+ c c c +, use the \verb+\\+ to go to a new line, and the \verb+&+ to switch between cells. 

\begin{center}
	\begin{tabular}{ c c c }
		cell1 & cell2 & cell3 \\ 
		cell4 & cell5 & cell6 \\  
		cell7 & cell8 & cell9    
	\end{tabular}
\end{center}


\subsection{Column lines}

Tabular usually allows more flexibility. Note that a \verb+|+ between the \verb+ c c c+ allows those vertical lines.

\begin{center}
	\begin{tabular}{ |c|c|c| } 
		\hline
		cell1 & cell2 & cell3 \\ 
		cell4 & cell5 & cell6 \\ 
		cell7 & cell8 & cell9 \\ 
		\hline
	\end{tabular}
\end{center}

\subsection{Additional Options}
The \verb+c+ doesn't mean "column," it means "center." To right- or left-align, use \verb+r+ or \verb+l+ instead.\\

To insert a horizontal line, use \verb+\hline+ where you want the line to be. The \verb+[0.5ex]+ are adding vertical space after the text.

\begin{center}
	\begin{tabular}{||c c c c||} 
		\hline
		Col1 & Col2 & Col2 & Col3 \\ [0.5ex] 
		\hline\hline
		1 & 6 & 87837 & 787 \\ 
		\hline
		2 & 7 & 78 & 5415 \\
		\hline
		3 & 545 & 778 & 7507 \\
		\hline
		4 & 545 & 18744 & 7560 \\
		\hline
		5 & 88 & 788 & 6344 \\ [1ex] 
		\hline
	\end{tabular}
\end{center}


\subsection{Fixing Lengths}
The following specification of column widths requires the \verb+array+ package.

\begin{center}
	\begin{tabular}{ | m{5em} | m{1cm}| m{1cm} | } 
		\hline
		cell1 dummy text dummy text dummy text& cell2 & cell3 \\ 
		\hline
		cell1 dummy text dummy text dummy text & cell5 & cell6 \\ 
		\hline
		cell7 & cell8 & cell9 \\ 
		\hline
	\end{tabular}
\end{center}


\subsection{Combining rows and cols}
More fancy stuff.

\begin{tabular}{ |p{3cm}||p{3cm}|p{3cm}|p{3cm}|  }
	\hline
	\multicolumn{4}{|c|}{Country List} \\
	\hline
	Country Name     or Area Name& ISO ALPHA 2 Code &ISO ALPHA 3 Code&ISO numeric Code\\
	\hline
	Afghanistan   & AF    &AFG&   004\\
	Aland Islands&   AX  & ALA   &248\\
	Albania &AL & ALB&  008\\
	Algeria    &DZ & DZA&  012\\
	American Samoa&   AS  & ASM&016\\
	Andorra& AD  & AND   &020\\
	Angola& AO  & AGO&024\\
	\hline
\end{tabular}


\subsection{Putting several rows in a cell with multirow}

\begin{center}
	\begin{tabular}{ |c|c|c|c| } 
		\hline
		col1 & col2 & col3 \\
		\hline
		\multirow{3}{4em}{Multiple row} & cell2 & cell3 \\ 
		& cell5 & cell6 \\ 
		& cell8 & cell9 \\ 
		\hline
	\end{tabular}
\end{center}

\subsection{Multi-page tables with longtable package}

For multiple-paged tables, you need another package.
	
	\begin{longtable}[c]{| c | c |}
		\caption{Long table caption.\label{long}}\\
		
		\hline
		\multicolumn{2}{| c |}{Begin of Table}\\
		\hline
		Something & something else\\
		\hline
		\endfirsthead
		
		\hline
		\multicolumn{2}{|c|}{Continuation of Table \ref{long}}\\
		\hline
		Something & something else\\
		\hline
		\endhead
		
		\hline
		\endfoot
		
		\hline
		\multicolumn{2}{| c |}{End of Table}\\
		\hline\hline
		\endlastfoot
		
		Lots of lines & like this\\
		Lots of lines & like this\\
		Lots of lines & like this\\
		Lots of lines & like this\\
		Lots of lines & like this\\
		Lots of lines & like this\\
		Lots of lines & like this\\
		Lots of lines & like this\\
		...
		Lots of lines & like this\\
	\end{longtable}


\subsection{Placing Tables}
You need the package \verb+float+ for this one, but it goes exactly where you specify:

Instead of \verb+h!+ as used below, you can also use \verb+h+ for approximately here, \verb+t+ for the top of the page; \verb+b+ for the bottom of the page; \verb+p+ in its own special page. \verb+!+ says to override latex's other commands, and \verb+H+ is basically \verb+h!+.
\begin{table}[h!]
	\centering
	\begin{tabular}{||c c c c||} 
		\hline
		Col1 & Col2 & Col2 & Col3 \\ [0.5ex] 
		\hline\hline
		1 & 6 & 87837 & 787 \\ 
		2 & 7 & 78 & 5415 \\
		3 & 545 & 778 & 7507 \\
		4 & 545 & 18744 & 7560 \\
		5 & 88 & 788 & 6344 \\ [1ex] 
		\hline
	\end{tabular}
\end{table}

\subsection{Cross Referencing}\label{table_cross_referencing}
Just like figures, tables can have captions, labels, and cross-referencing that Latex takes care of for you as long as you remember to label intelligently.

The table \ref{table:1} is an example of referenced \LaTeX elements.

\begin{table}[h!]
	\centering
	\begin{tabular}{||c c c c||} 
		\hline
		Col1 & Col2 & Col2 & Col3 \\ [0.5ex] 
		\hline\hline
		1 & 6 & 87837 & 787 \\ 
		2 & 7 & 78 & 5415 \\
		3 & 545 & 778 & 7507 \\
		4 & 545 & 18744 & 7560 \\
		5 & 88 & 788 & 6344 \\ [1ex] 
		\hline
	\end{tabular}
	\caption{Table to test captions and labels}
	\label{table:1}
\end{table}

\subsection{Documenting Tables}
The \verb+front_matter_template+ file includes the command \verb+\listoftables+ which prints a page that lists the tables you've created.

\subsection{Other }
See \url{https://www.overleaf.com/learn/latex/tables} for some more customization on coloring cells and such.
