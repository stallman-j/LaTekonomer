\section{Fancier Tables: Threeparttable}\label{sec:threeparttable}


Making highly editable tables can be a bit of a hassle. To have a table with notes, you'll want the package \verb+threepartable+. 

For a table that spans multiple pages, you need \verb+longtable+. 

For a table that spans multiple pages AND has customizable notes, you'll want to add the package \verb+threeparttablex+, which allows you to use \verb+longtable+ with \verb+threeparttable+. You'll need to spend a little time finagling with the table notes in order to get this to work. See \verb+longtable-with-text-wrapping+ (if it's included, it's in \ref{sec:longtable-with-text-wrapping}) for longtable examples.

(to be continued)
\subsection{threeparttable}
See \href{https://tug.org/pracjourn/2007-2/asknelly/}{here} for some examples.
The following is an example from \href{https://tex.stackexchange.com/questions/118743/threeparttable-notes-layout}{tex stack exchange}


\begin{table}[htp]
\caption{Some very informative caption}
\begin{center}
\begin{threeparttable}
\begin{tabular}{c c c c}
    \toprule
    \textbf{1st Column} & \textbf{2nd Colimn} & \textbf{3rd Colimn} & \textbf{4th Colimn} \\ \midrule
      QWERTY\tnote{1}   &                     &                     &  \\
      ASDFGH\tnote{2}   &                     &                     &  \\ \bottomrule
\end{tabular}
\begin{tablenotes}
\item[1] qwerty; \item[2] asdfgh
\end{tablenotes}
\end{threeparttable}
\end{center}
\label{table:simDisimCoefNewDef}
\end{table}



\begin{table}[H]
\centering
\caption{...}
\footnotesize
{
\def\sym#1{\ifmmode^{#1}\else\(^{#1}\)\fi}
\begin{threeparttable}

\begin{tabular}{c c c c}
    \toprule
    \textbf{1st Column} & \textbf{2nd Colimn} & \textbf{3rd Colimn}\tnote{\dag} & \textbf{4th Colimn} \\ \midrule
      QWERTY\tnote{1}   &                     &                     &  \\
      ASDFGH\tnote{2}   &                     &                     &  \\ \bottomrule
\end{tabular}
  \begin{tablenotes}
    \item[$*$] $p<0.1$, \sym{**} $p<0.05$, \sym{***} $p<0.01$
    \item[\dag] These ... \smallskip
    \item \emph{Note:} This is a note. I'm not sure if it will switch onto the next line as I'd like, so here I am going to go ahead and keep writing to see if it will indeed switch lines
  \end{tablenotes}

\end{threeparttable}
}
\label{tbl:name}
\end{table}



   