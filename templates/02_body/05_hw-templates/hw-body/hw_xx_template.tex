\noindent
\Large SDS x17  
 	\large Problem Set 03  
 	\large Due April 11, 2022 
 	\large Jillian Stallman 
 	\today  [1.5em] % today's date, then gives a little space below
 	

I'd like to express my deep gratitude to Rachel Renne, Seung Min Kim, Weining Fang, and Yihong Wang for collaboration and support. 

My code assumes that you have installed the \verb+causalToolbox+ and \verb+rForestry+ and some other packages that are common.

\section{Problem 1 Transphobia 1}


\subsection{Pre-Processing}

    Given that door-to-door canvassing is not an area of expertise for me, I follow the authors' rationales as indicated in their supplementary materials to select outcome variables \citep[p.6-10]{broockmanSupplementaryMaterialsDurably2016}. The transphobia literature at this time lacked widely accepted measures to measure anti-transgender prejudice, so the authors constructed several indices in an attempt to measure negative attitudes more broadly. 
    
   
    
    Consider \verb+gender\_norm\_moral\_t\#+  I find the \verb+gender\_norm\_looks\_t\#+ to be an interesting question: 
    \begin{quote}
    To keep children from being confused, it's better when men look and act like men, and women look and act like women.    
    \end{quote} 
    
    
    \textbf{Covariates to Include:}
    
    The authors describe their preferred covariates in their Supplementary Materials, although I don't necessarily share their priors. They suggest that Democrats will be more responsive to treatment than Republicans:
    \begin{itemize}
        \item \verb+vf\_racename+ (possibly different cultural experiences or personal experience with different gender expression, and variation in social desirability of certain views)
        \item \verb+vf\_party+ (may pick up differences in social desirability)
    \end{itemize}
    
    \textbf{Preliminary Data Processing:}
    Figures~\ref{fig:hw-04_q-01a_data-skim_keepvars-01} and ~\ref{fig:hw-04_q-01a_data-skim_keepvars-02} present an initial look at the data with \verb+skim+. I have included additional outcomes for the purposes of missing data analysis, to which we now turn. 
    

    \begin{figure}[H]
	\centering % centers
	\includegraphics[width=.8\linewidth]{figures/frog.jpg}
	\caption{An Initial Look at the Data} 
	\label{fig:hw-04_q-01a_data-skim_keepvars-01}
    \end{figure}
    
        \begin{figure}[H]
	\centering % centers
	\includegraphics[width=.8\linewidth]{figures/frog.jpg}
	\caption{An Initial Look at the Data} 
	\label{fig:hw-04_q-01a_data-skim_keepvars-02}
    \end{figure}

   
            

            &        mean&          sd&         min&         max&       count\\
\midrule
Apple Valley&          29&       13.07&           6&         100&       1,072\\
Blaine      &          30&       14.48&           6&         102&       1,068\\
Brainerd    &          24&       12.04&           6&         109&       1,070\\
Detroit Lakes&          21&       12.39&           5&          88&       1,076\\
Ely         &          17&        9.68&           5&          95&       1,066\\
Fond Du Lac &          21&       11.77&           5&         141&       1,074\\
Grand Portage&          20&        8.86&           5&          65&         892\\
Lakeville   Near Road&          26&       12.68&           6&          97&       1,088\\
Leech Lake Nation Cass Lake&          28&       14.04&           9&         100&         363\\
Marshall    &          22&       11.47&           6&         109&       1,070\\
Marshall Terrace&          31&       13.60&           8&          77&         449\\
Minneapolis Phillips&          31&       14.06&           7&          94&       1,068\\
Red Lake Nation&          24&       11.96&           5&         105&       1,005\\
Rochester   &          30&       13.44&           5&          98&       1,059\\
Shakopee    &          25&       16.13&           7&          91&          98\\
St. Cloud   &          25&       12.92&           5&          90&       1,085\\
St. Michael &          27&       13.10&           6&          89&       1,083\\
St. Paul Harding H.S.&          30&       14.21&           6&          92&       1,072\\
Virginia    &          20&       11.66&           5&         107&       1,085\\
West Duluth &          22&       12.53&           5&          94&       1,067\\
Total       &          25&       13.25&           5&         141&      18,910\\


   
    

    \begin{figure}
    \centering
    \begin{minipage}{0.45\textwidth}
        \centering
        \includegraphics[width=\textwidth]{figures/frog.jpg} % first figure itself
        \caption{Forest as Ntree grows}
       \label{fig:hw-04_q-01e_forest}
    \end{minipage}\hfill
    \begin{minipage}{0.45\textwidth}
        \centering
        \includegraphics[width=\textwidth]{figures/frog.jpg} % second figure itself
        \caption{Xgboost as Nrounds Grows}
    \end{minipage}
    \label{fig:hw-04_q-01e_xg}
\end{figure}






\section{Asymptotic Properties of the AIPW Estimator}

    Please see the attached pages for my work regarding these problems. 
    
    \includepdf[pages={1-9}]{included-pdfs/hw-04/sds-x17_hw-04_stallman_q-03.pdf} 

