\documentclass{article} % can also have "report", "book", and others. Note that "article" doesn't accept chapters.

% Packages:
%% Language and font encodings
\usepackage[english]{babel}
\usepackage[utf8x]{inputenc}
\usepackage[T1]{fontenc}
\usepackage{palatino}

%% Page-Format-Related Packages
\usepackage{lscape} % allows you to have landscape pages in a portrait document
% Load the parskip package with skip and indent options
\usepackage{pdflscape} % for tables to be landscape

\usepackage[skip=10pt plus1pt, indent=15pt]{parskip}

%% Sets page size and margins
\usepackage[a4paper,top=3cm,bottom=2cm,left=3cm,right=3cm,marginparwidth=1.75cm,margin=1in]{geometry}
\linespread{1.2} %1.2 spacing


%% Graphics Packages
\usepackage{graphicx,pstricks}
\usepackage{graphics}
\usepackage{float} % allows [H] to specify the exact location of graphics
\usepackage{subfigure}
\graphicspath{{figures/}} % tells latex that pics are in the figures folder
\usepackage{verbatim} % inputting directly verbatim
\usepackage{listings} % input from txt
\lstset{
basicstyle=\scriptsize\tt,
}

%% Tables Packages
\usepackage{dcolumn} % aligns decimals in table columns
\usepackage{booktabs} % Common for publishing-ready tables, necessary.
\usepackage{tabularx} % more table customization, also necessary if you want to change column size
\usepackage{array} % Necessary for general but pretty tables.
\usepackage{threeparttable} % allows for captions and notes to be the same width. 
\usepackage{multirow} % another customization package
\usepackage{longtable} % for multi-page tables
\usepackage{threeparttablex} % for longtable plus threeparttable. Note that notes formatting is finicky.



%% Math-Related Packages
\usepackage{amssymb,amsmath,amsthm}  % math environments
%\usepackage[mathscr]{eucal}
\DeclareMathOperator*{\argmax}{arg\,max} % define a particular math command
\DeclareMathOperator*{\argmin}{arg\,min}
\newtheorem{theorem}{Theorem}[section] % allows you to say \begin{theorem} and it numbers them accordingly, restarting at each new section.
\newtheorem{corollary}{Corollary}[theorem] % restarts the counter of corollary with each new theorem
\newtheorem{lemma}[theorem]{Lemma} % uses the same counter as the "theorem" environment
\newtheorem{assumption}{Assumption}[section] % restarts the counter with each new section
\usepackage[nosolutionfiles]{answers} % also allows for putting solutions within solutions say for an answer key, can print 2 versions
\Newassociation{sol}{Solution}{ans}
\newtheorem*{definition}{Definition}
\newtheoremstyle{mystyle}% name
  {\topsep}% Space above
  {\topsep}% Space below
  {\normalfont}% Body font
  {}% Indent amount
  {\bfseries}% Theorem head font
  {}%Punctuation after theorem head
  {.5em}%Space after theorem head
  {}% theorem head spec
\theoremstyle{mystyle}
\newtheorem{ex}{}
\newenvironment{Exercise}[1]
{\item #1
\begin{enumerate}
}
{
\end{enumerate}
}

\newcommand{\Question}[1]{%
\item #1
}


\newtheorem{hyp}{Hypothesis}[section] % check this, should be that it starts over the counter with each new section
\usepackage{thmtools} % used to customize theorem styles
\declaretheoremstyle[
spaceabove=6pt, spacebelow=6pt,
headfont=\normalfont\bfseries,
notefont=\mdseries, notebraces={(}{)},
bodyfont=\normalfont,
postheadspace=0.6em,
headpunct=:
]{hypstyle}
%\declaretheorem[style=hypstyle, name=Hypothesis, preheadhook={\renewcommand{\thehyp}{(H\arabic{hyp})}}]{hyp}

\usepackage{pdfpages}



% Cross-Referencing Packages
\usepackage{xurl} % specifically created to allow urls to break across lines
\usepackage[hidelinks]{hyperref} % if you're using this, remember to mention to Jillian and she'll get you set up. hidelinks option makes it so there isn't an annoying colored box around the links. Pro tip that's what noobs always miss.
\usepackage{cleveref} % used for particularly fancy cross-referencing styles. %% IMPORTANT: make sure cleverref is the LAST cross-referencing package you load, otherwise it can cause some conflicts.
\crefname{hyp}{hypothesis}{hypotheses}
\Crefname{hyp}{Hypothesis}{Hypotheses}



% Environment Packages
\usepackage{enumitem} %nice for making lists/outlines from scratch.
\usepackage{tasks} % for allowing HW questions in multiple columns and rows
\usepackage{comment} % enables blocking out comments
\usepackage[toc]{appendix} % for appendices. toc adds appendix to TOC and [page] puts a page in between the text and it, common options [toc,page]
%\renewcommand{\appendixname}{Annex} % if you wanted to rename what the appendix was called



% Bibliography Packages
\usepackage{natbib} % this is one of the citation-specific packages
\bibliographystyle{apalike}
\setcitestyle{authoryear,open={(},close={)}}

%\bibliographystyle{aomart}
%\bibliographystyle{ksfh_nat}
%\setcitestyle{authoryear,open={(},close={)}}
%\bibliographystyle{IEEEbib}






% Title page
\title {Your Title Goes Here}
\author {John Doe}
\date{\today}

%-------------------------------------------------------------%
% This is the master document that houses the homework solutions


% Last edited: 2022-11-29 by Jillian Stallman
% ------------------------------------------------------------%




\begin{document}




% ------%
% Template
% ------%

% new homework solutions template
%\noindent
\Large ECON 330 / EVST 340 / ECON 737 / ENV 804 \\
 	\large Problem Set 04 \\
 	\large Due October 04, 2021 at 10:30 a.m.\\
 	\large Jillian Stallman\\
 	\today \\[1.5em] % today's date, then gives a little space below
 	
 	
 \section*{Solutions (Short)}
\begin{enumerate}
    \item S
    \begin{enumerate}
        \item h
        \item The shado
    \end{enumerate}
    
    \item T
    \begin{enumerate}
        \item T
        \item jh
        \item If 
    \end{enumerate}
    \item The
    \begin{enumerate}
        \item Now 
        \item i
    \end{enumerate}
\end{enumerate}


\Opensolutionfile{ans}[ans1]


\begin{enumerate}
\begin{Exercise}{Calculate the following:}
    \Question{$7+2$}
     \begin{sol}
         \textbf{Answer:} $9$
     \end{sol}
     
     
    \Question{$9-9$}
      \begin{sol}
         \textbf{Answer:}  $0$
      \end{sol}
\end{Exercise}

\begin{Exercise}{Solve the following equations:}
    \Question{$x+5=7$}
      \begin{sol}
       \textbf{Answer:}  $x=2$
     \end{sol}
     
     
    \Question{$x-5=9$}
    \begin{sol}
    \textbf{Answer:} $x=14$
    \end{sol}
    
    
    \Question{$5x=20$}
    \begin{sol}
    \textbf{Answer:} $x=4$
         
    \end{sol}
    

    

\end{Exercise}



\begin{Exercise}{Calculate the following:}
    \Question{$7+2$}
     \begin{sol}
         \textbf{Answer:} $9$
     \end{sol}
     
     \end{Exercise}

\end{enumerate}

\Closesolutionfile{ans}


    \begin{figure}[H]
	\centering % centers
	\includegraphics[width=.8\linewidth]{figures/hw-01/hw-01-mcmb-tax.jpg}
	\caption{Achieving Optimal Net Benefits with Efficient Tax} % note caption is required in order to get the cross referencing
	\label{fig:ps01efficienttax}
    \end{figure}

    Figure~\ref{fig:ps01efficienttax} refers to the figure above.

 \begin{figure}
        \centering
        \includegraphics[width = .5 \textwidth]{current-events/nonrenewable-resources/02_natural-gas-wells/diversified-energy-gas-wells.png}
        \begin{minipage}{.5\textwidth}
         \footnotesize Source: \href{https://www.bloomberg.com/features/diversified-energy-natural-gas-wells-methane-leaks-2021/?sref=1kJVNqnU}{Bloomberg 2021}
        \end{minipage} 
    \end{figure}



% ----------------------------%
% Problem Set Files
% ----------------------------%


% --------------%
% First-round Solutions
% --------------%
% Currently revising the following file:
\noindent
\Large SDS x17  
 	\large Problem Set 03  
 	\large Due April 11, 2022 
 	\large Jillian Stallman 
 	\today  [1.5em] % today's date, then gives a little space below
 	

I'd like to express my deep gratitude to Rachel Renne, Seung Min Kim, Weining Fang, and Yihong Wang for collaboration and support. 

My code assumes that you have installed the \verb+causalToolbox+ and \verb+rForestry+ and some other packages that are common.

\section{Problem 1 Transphobia 1}


\subsection{Pre-Processing}

    Given that door-to-door canvassing is not an area of expertise for me, I follow the authors' rationales as indicated in their supplementary materials to select outcome variables \citep[p.6-10]{broockmanSupplementaryMaterialsDurably2016}. The transphobia literature at this time lacked widely accepted measures to measure anti-transgender prejudice, so the authors constructed several indices in an attempt to measure negative attitudes more broadly. 
    
   
    
    Consider \verb+gender\_norm\_moral\_t\#+  I find the \verb+gender\_norm\_looks\_t\#+ to be an interesting question: 
    \begin{quote}
    To keep children from being confused, it's better when men look and act like men, and women look and act like women.    
    \end{quote} 
    
    
    \textbf{Covariates to Include:}
    
    The authors describe their preferred covariates in their Supplementary Materials, although I don't necessarily share their priors. They suggest that Democrats will be more responsive to treatment than Republicans:
    \begin{itemize}
        \item \verb+vf\_racename+ (possibly different cultural experiences or personal experience with different gender expression, and variation in social desirability of certain views)
        \item \verb+vf\_party+ (may pick up differences in social desirability)
    \end{itemize}
    
    \textbf{Preliminary Data Processing:}
    Figures~\ref{fig:hw-04_q-01a_data-skim_keepvars-01} and ~\ref{fig:hw-04_q-01a_data-skim_keepvars-02} present an initial look at the data with \verb+skim+. I have included additional outcomes for the purposes of missing data analysis, to which we now turn. 
    

    \begin{figure}[H]
	\centering % centers
	\includegraphics[width=.8\linewidth]{figures/frog.jpg}
	\caption{An Initial Look at the Data} 
	\label{fig:hw-04_q-01a_data-skim_keepvars-01}
    \end{figure}
    
        \begin{figure}[H]
	\centering % centers
	\includegraphics[width=.8\linewidth]{figures/frog.jpg}
	\caption{An Initial Look at the Data} 
	\label{fig:hw-04_q-01a_data-skim_keepvars-02}
    \end{figure}

   
            

            &        mean&          sd&         min&         max&       count\\
\midrule
Apple Valley&          29&       13.07&           6&         100&       1,072\\
Blaine      &          30&       14.48&           6&         102&       1,068\\
Brainerd    &          24&       12.04&           6&         109&       1,070\\
Detroit Lakes&          21&       12.39&           5&          88&       1,076\\
Ely         &          17&        9.68&           5&          95&       1,066\\
Fond Du Lac &          21&       11.77&           5&         141&       1,074\\
Grand Portage&          20&        8.86&           5&          65&         892\\
Lakeville   Near Road&          26&       12.68&           6&          97&       1,088\\
Leech Lake Nation Cass Lake&          28&       14.04&           9&         100&         363\\
Marshall    &          22&       11.47&           6&         109&       1,070\\
Marshall Terrace&          31&       13.60&           8&          77&         449\\
Minneapolis Phillips&          31&       14.06&           7&          94&       1,068\\
Red Lake Nation&          24&       11.96&           5&         105&       1,005\\
Rochester   &          30&       13.44&           5&          98&       1,059\\
Shakopee    &          25&       16.13&           7&          91&          98\\
St. Cloud   &          25&       12.92&           5&          90&       1,085\\
St. Michael &          27&       13.10&           6&          89&       1,083\\
St. Paul Harding H.S.&          30&       14.21&           6&          92&       1,072\\
Virginia    &          20&       11.66&           5&         107&       1,085\\
West Duluth &          22&       12.53&           5&          94&       1,067\\
Total       &          25&       13.25&           5&         141&      18,910\\


   
    

    \begin{figure}
    \centering
    \begin{minipage}{0.45\textwidth}
        \centering
        \includegraphics[width=\textwidth]{figures/frog.jpg} % first figure itself
        \caption{Forest as Ntree grows}
       \label{fig:hw-04_q-01e_forest}
    \end{minipage}\hfill
    \begin{minipage}{0.45\textwidth}
        \centering
        \includegraphics[width=\textwidth]{figures/frog.jpg} % second figure itself
        \caption{Xgboost as Nrounds Grows}
    \end{minipage}
    \label{fig:hw-04_q-01e_xg}
\end{figure}






\section{Asymptotic Properties of the AIPW Estimator}

    Please see the attached pages for my work regarding these problems. 
    
    \includepdf[pages={1-9}]{included-pdfs/hw-04/sds-x17_hw-04_stallman_q-03.pdf} 



% --------------%
% Template Solutions
% --------------%
% Currently revising the following file:


% Past solutions to be commented out
\begin{comment} % comment out past solutions
\include{hw-solution-guides/hw-01/hw-01_solutions_final}
\noindent
\Large ECON XX \\
 	\large Problem Set 05 \\
 	\large Due October 18, 2021 at 10:30 a.m.\\
 	\large Solutions \\
 	\today \\[1.5em] % today's date, then gives a little space below

\section*{Solutions (short)}
    \begin{enumerate}
        \item Question 1
        \begin{enumerate}
            
        \item Part a
        \end{enumerate}
        
        \item Question 2
        \begin{enumerate}
            \item Part b
        \end{enumerate}
    \end{enumerate}

\section*{Solutions (long)}

\Opensolutionfile{ans}[ans1]

\begin{enumerate}

\begin{Exercise}{Suppose you have an initial stock of oil equal to 6,730 Quads, a stock of natural gas equal to 7,740 Quads, and a stock of coal equal to 20,300 Quads of energy. Suppose that the cost of extracting oil is \$1.5 million/Quad, the cost of extracting natural gas is \$2 million / Quad, and the cost of extracting coal is \$3 million / Quad.\\

The annual global demand $Q$ for energy in Quads is 

    \begin{equation}
        Q(P) = 13,700,000P^{-0.7},
        \label{eq:demandenergy}
    \end{equation}
    where $P$ is price. The market interest rate $r$ is 4\%. There is a ceiling price for renewable energy at \$100 million / Quad. Suppose you can use these resources one at a time.\\
    
    Create an Excel file to answer the following questions, solving backwards through time.
    :}
    

    
    
    \Question{Question question?}
     \begin{sol}
         \textbf{Answer:} Answer here.
         
     \end{sol}
    

    
    \Question{Another question?}

      \begin{sol}
       \textbf{Answer:}\\
       Another answer
     \end{sol}
     
     
    \Question{Another question?}
    \begin{sol}
    \textbf{Answer:}\\
    
    Some equations
    
        \begin{equation}
            \frac{\dot{P(t)}}{P(t)}=r\left(1-\frac{C}{P(t)}\right),
            \label{eq:opt-time-der}
        \end{equation}
        
    or rearranging
    
        \begin{equation*}
            \dot{P(t)}=P(t)r\left(1-\frac{C}{P(t)}\right).
        \end{equation*}
    
    
    
    
    
    \end{sol}
    
    
    \Question{Another question}
    \begin{sol}
    \textbf{Answer:} Split equations
    
        \begin{equation}
        \begin{split}
            \max_{ \{Q(t)\}_{t=0}^\infty} & \int_{0}^\infty \left[P(t)-C\right] Q(t) e^{-rt}dt\\
            \text{such that } & \int_{t=0}^\infty Q(t) = S
        \end{split}
        \label{eq:npv}
        \end{equation}
        
        A list with equations
        
        \begin{enumerate}
            \item item with equation
            \begin{equation*}
                P(0)-C = [P(t)-C]e^{-rt}.
            \end{equation*}
            \item Time horizon:
            \begin{equation*}
                \int_{t=0}^\infty Q(t) = S
            \end{equation*}
            \item final comment
        \end{enumerate}
        
        \textbf{Note:}\\
        
        Another note
        
        \textbf{The Implementation:}\\
        
                \noindent 
                \textbf{Step 1:}\\        
        We can rearrange Equation~\eqref{eq:opt-time-der}:
        
        \begin{equation*}
            \dot{P(t)}=P(t)r\left(1-\frac{C_{coal}}{P(t)}\right)
        \end{equation*}
        
        
        Here's a way to write verbatim. Column \texttt{verbatim(t) think}, the demand \textit{in} time $t$.

        Another verbatim thing if we want to talk about spreadsheets.
        
        \begin{verbatim}
            =SUM($D$3:D3)
        \end{verbatim}
        and drag down.\\
        
        The \texttt{\$D\$3} dollar signs say that this reference doesn't change (it's the absolute reference. You can also use these double dollar signs for the cells where you pull the interest rates and extraction costs from), but the \texttt{D3} will change as you go down the column. For $T-2$, you should have 71.75; for $T-198$, you get 50,548, which is clearly greater than the stock of 20,300. We see that at $T-118$ is where we go from 20,025 to $T-119$ at 20,367.\\
        
        Alternatively, you could dispense with the running sum, start from the stock of 20,300, and subtract backwards. If you subtracted backwards, it goes from a positive 274 to negative 67 at this same time, so we would make the transition sometime in that year.\\
        
        If we say we made the switch in $T-118$ rather than $T-119$, the implication is that we would rather leave some of the coal in the ground than extract more coal than exists on the planet. This seems more reasonable. \\
        
        \textbf{Step 3: Price and Quantity for Gas}\\
        
        We know that we'll make the switch at time $T-118$, at which point we're now going to be on the price path for gas, rather than coal. Our cost of extraction is different. Now we have $C_{gas}=2$ million, rather than the \$3 million for coal. Equation~\eqref{eq:pricepathcoal} becomes
        
        \begin{equation}
            \begin{split}
            P(T-1) & = P(T)-r(P(T)-C_{gas}).
            \end{split}
            \label{eq:pricepathgas}
        \end{equation}       
        
        At $T-119$, then, we take
        
        \begin{equation*}
            P(T-119) = P(T-118)-r(P(T-118)-C_{gas}).
        \end{equation*}
        
        Here $P(T-118)$ is calculated based on the coal price. We can make another column \texttt{Price Gas} to store this information, starting the price path now at $T-119$ rather than $T$. The price makes a jump from $P(T-118)=3.78$ million to $P(T-119)=3.71$ million. For $P(T-120)$, remember to re-enter the formula it so that you're taking $P(T-119)$ from the \textit{gas} column. If you drag down from $P(T-119)$ without making this adjustment, you'll get the wrong price sequence for gas. $P(T-198)$ should get you 2.07 million.\\
        
        Do the same thing with calculating the demand and cumulative demand for gas as we did for coal. We find that we exhaust the gas pretty quickly (there wasn't much stock relative to coal), in $T-138$.\\
        
        We still have the question of what we do with the little extra bit of stock of coal left over. There are a couple assumptions you could make that would change the details but won't change your final calculations much. For instance, you could assume that the leftover stock is wasted, which would be like saying that the first bit of, say, coal that you use in the year you've used up the natural gas just gets burned.\\
        
        Another slightly more realistic assumption might be to say that, in the year in which the transition is made from gas to coal, the first little quantity of coal is so easy to extract that it's effectively the same marginal cost as the coal. This gives us a little buffer between the gas and the coal, and makes slightly more sense than saying the resource is burned, but either assumption could be justified, and we're after all making a discrete approximation to a continuous process. In the excel sheets, we've assumed that the leftover coal becomes effectively a natural gas stock.\\
        
        \textbf{Step 4: Price and Quantity for Oil}\\
        
        Now let's do the same thing for oil. We should get $P(T-139)=2.74$ million. Depending on the assumption you make about extra stocks of gas and coal, the stock of oil goes to zero sometime between $T-152$ and $T-153$. This means that somewhere in this time is where time, going forwards, actually starts. We exhaust the resources about 153 years. \\
        
        Tracking backward, then, we can get that the price of energy is \$2.2 million.
        
        
        
        
        

        
        
    \end{sol}
    
    
    \Question{What is the net present value of each resource to society?}
    \begin{sol}
    \textbf{Answer:}\\
    
    We need to translate our integral net present value formula from Equation~\ref{eq:npv} into a discrete net present value. We can do this by summing up over the years the net present value of the stock for each year. For instance, in $T-1$, which is the 152nd year, we use coal, and we'll get\footnote{See Appendix~\ref{sec:npvintegrals} for a little intuition about this net present value and how it relates to the integral we see in class.}
    
    \begin{equation}
        NPV(Coal_{t=152}) = [P(152) - C]Q(152)e^{-r(152)}.
        \label{eq:npvcoal}
    \end{equation}
    
    
    
    
    We'll get the total net present value if we sum these similar expressions up over all times in which we're extracting coal, call it $NPV(Coal_{t=start}^{t=end})$
    
    \begin{equation*}
        NPV(Coal_{t=118}^{t=152}) = \sum_{t=118}^{152} [P(t)-C]Q(t)e^{-rt}.
    \end{equation*}
    
    In general, for any of the resources, we can write for $Resources$ and $t=$start the time to start extraction, $t=$end the final year of extraction:
    
    \begin{equation*}
        NPV(\text{Resource}_{t=\text{start}}^{t=\text{end}}) = \sum_{t=\text{start}}^{\text{end}} [P(t)-C]Q(t)e^{-rt}.
        \label{eq:npvall}
    \end{equation*}
    
    In excel, we can make a column for time running forwards, from $t=153$ at the top to $t=0$ at the bottom. We then calculate the net present value for a particular year in another column, and tally them up in a third column.\\
    
    This gives a total of \$4 billion.\\
    
    Doing the same for gas gives us \$3.36 billion, and for oil \$4.89 billion.
    

    
    
    \end{sol}
    

    

\end{Exercise}


\begin{Exercise}{Suppose that the environmental damage from burning oil is \$23.5 million/ Quad, natural gas is \$25 million / Quad and the environmental damage from burning coal is \$32 million / Quad.}
    \Question{How does that change the optimal answers to Question 1?}
     \begin{sol}
         \textbf{Answer:}\\
         
         We'll perform the same exercise as before, but incorporating the environmental damage as an additional cost. Lecture 11, slide 11 and the visual with slide 12 gives us the relevant information. Our problem now becomes
         
        \begin{equation}
        \begin{split}
            \max_{ \{Q(t)\}_{t=0}^\infty} & \int_{0}^\infty \left[P(t)-C-E(t)\right] Q(t) e^{-rt}dt\\
            \text{such that } & \int_{t=0}^\infty Q(t) = S
        \end{split}
        \label{eq:npv-damage}
        \end{equation}
        
        Optimality conditions now become:
        
        \begin{enumerate}
            \item The net price should be the same for all periods. That is, for all $t$ the following should hold:
            \begin{equation*}
                P(0)-C = [P(t)-C-E(t)]e^{-rt}
            \end{equation*}
            or
            
            \begin{equation}
        \dot{P} = P(t)r\left(1-\frac{C+E(t)}{P(t)}\right)
        \label{eq:opt-envidamage}
            \end{equation}
            \item The resources should be consumed over the time horizon:
            \begin{equation*}
                \int_{t=0}^\infty Q(t) = S
            \end{equation*}
            \item Any resource whose cost exceeds the price shouldn't be extracted.
        \end{enumerate}
        
        As noted in the slides, we should find 1) a higher initial price, and 2) a flatter price path, dragging out the time of exploitation of the fuels.\\
        
        Therefore, if we find that our time is \textit{less} than 153 years, or if we find that the initial price  is \textit{less} than 2.2 million, something has gone wrong.\\
        
        The intuition for taking longer is that the inclusion of the marginal damages acts like a cost. Just like we want to defer costs into the future so that their present value is smaller, we also want to defer the damages. This means sending more of the costs and damages into the future, which lengthens the time of extraction.\\
        
        Because the coal is both the dirtiest and the most expensive to extract, we still extract it last (the cumulative cost is still greatest). Ditto for natural gas and for oil. Therefore our order of extraction remains the same: oil, natural gas, and coal.\\
        
        If the environmental damages were dispersed differently, that might change the order, but it would depend on the relative magnitudes of costs of extraction and environmental damages.
        

        
        
     \end{sol}

    \Question{What's the final price of energy when the fossil fuels are exhausted?}
     \begin{sol}
         \textbf{Answer:}\\
        \$100 million / Quad. The ceiling price is the same, so prices will end up at the same place. (Try to run through the logic of why it can't be higher or lower than \$100 million again without flipping back to the previous question to see if you really get where it's coming from).     
    \end{sol}
    
    
    \Question{How fast does the price of coal increase during its extraction?}
     \begin{sol}
         \textbf{Answer:}\\
         Our optimality condition is now Equation~\eqref{eq:opt-envidamage}, or rearranging
         
            \begin{equation}
        \dot{P} = r\left(P(t)-C-E(t)\right).
        \label{eq:dotp-envidamage}
            \end{equation}
            
      
            
            
    \end{sol}    

    \Question{What is the initial price of energy?}
     \begin{sol}
         \textbf{Answer:}\\
        
        As before, in order to figure out the initial price of energy, we'll need to work backwards. This time, our discrete approximation of Equation~\eqref{eq:dotp-envidamage} for the change in price will be
        
             \begin{equation*}
        P(t) - P(t-1) = r\left(P(t)-C-E(t)\right).
            \end{equation*}
            
        Rearranging for $P(t-1)$ gives how we'll get our approximation within each cell of our new spreadsheet:
        
        \begin{equation}
             P(t-1) = P(t)- r\left(P(t)-C-E(t)\right).
             \label{eq:approx-envidamages}
        \end{equation}  
        
        With much greater environmental damages than the marginal costs we'd seen previously, we'll end up using the resource at a \textit{much} slower pace.\\
        
        We can set this question up in another sheet, adding the environmental damages to the top row as parameters, and then calculating first the price of coal and demand for coal given that price, and then calculating the cumulative demand for coal.\\
        
        We get that at $k=301$, we still have positive stock of 70 left, but if $k=302$ then we have demanded more cumulative coal than there exists in the stock, so we make the transition in $T-301$.\\
        
        We'll again make the assumption that in that year of transition, there are 70 units of coal that are easy to extract at the same marginal cost of the resource we just transitioned from, which will be the natural gas.\\
        
        When we do this and calculate where we should make the transition, this little extra leads us to different year choices. If we add up the cumulative demand, we get that the stock of gas if exhausted in $T-395$. If we use the assumption that in the year the gas is exhausted, we start on a little bit of the coal, we get that we should transition in $T-396$. For the sake of consistency, we'll use $T-396$.\\
        
        We then get that the oil traces back to $T-471$, with an initial energy price of \$25.1 million, which is a much greater initial price than the \$2.2 million we'd found previously.
        
        
        
    \end{sol} 
    
    \Question{What is the net present value of each resource to society?}
     \begin{sol}
         \textbf{Answer:}\\
          When we do the net present value exercise as before, we should take now instead of Equation~\eqref{eq:npvall}, we now have

    \begin{equation}
        NPV(\text{Resource}_{t=\text{start}}^{t=\text{end}}) = \sum_{t=\text{start}}^{\text{end}} [P(t)-C-E(t)]Q(t)e^{-rt}.
        \label{eq:npvalldamages}
    \end{equation}
    
    We include the environmental damages because those damages are part of what we're discounting, that's why this problem differs from question 1. If we'd used just $P(t)-C$ inside the parentheses in Equation~\eqref{eq:npvalldamages}, this would have implied that we'd taken into account the damages when we made the choice to use stock, but somehow that we didn't take into account the damages when we valued them, which would be an inconsistent way to assess the value of the resources.\\
    
    When we do as in Question 1 for each of our three resources, we get for coal an NPV of \$7,603, a massive reduction from our previous \$40 million. When we take into account the damages that are accruing to the environment in our calculation, this resource stream is much less valuable.\\
    
    Likewise the net present value of the stream of profits from gas is \$66.8 million, relative to over \$3 billion in the case in which we didn't take into account the environmental cost.\\
    
    Finally, the net present value for oil is now \$699 million, compared with \$4.89 billion previously.
          
    \end{sol} 
    
\end{Exercise}

\end{enumerate}

\Closesolutionfile{ans}


\begin{appendices}


\section{Mathematical Notes}

\begin{subappendices}
\subsection{Hotelling's Rule Intuition}\label{sec:hotellingrule}
What's the intuition for this? 
    
    We know from our Hotelling rule (Lecture 10) that in a competitive market with no extraction cost, the price of the nonrenewable resource should rise at the rate of interest $r$, and it's easier to think of the intuition from this first.\\

    
    In that case, we have that optimality requires for flows
    
        \begin{equation*}
            \frac{\dot{P(t)}}{P(t)} = r
        \end{equation*}
    which in terms of stocks means
    
        \begin{equation*}
            P(t)=P(0)e^{rt}.
        \end{equation*}
    
    Why is this the case? We can consider the Hotelling rule as an arbitrage rule, and the nonrenewable resource as an investment opportunity for the firm. If the firm digs the coal out of the ground (here at cost zero), it can sell it on the market at the going price. Alternatively, it can decide \textit{not} to sell, and sell in the future, which is the "hoarding" we've discussed in class.\\
    
    One way to consider this is to say, suppose we have a price today that we think is the equilibrium price for coal today. We'll show that with a price path that's anything different than $r$, this proposed equilibrium price can't hold. It will either be too high or too low. The only way today's price will be stable is if the price path follows the interest rate.\\
    
    Suppose for the sake of argument that the price of coal increased at the rate of 2\%, which is below our rate of interest of 4\%. If this were the case, then it would make more sense for the firm to extract more today and put the money in the bank to earn 4\% interest for the future. This would push up the supply today, which would push down the price today. Therefore that price today can't be an equilibrium price.\\
    
    If on the other hand the price of coal was increasing at a rate of 10\%, which is above the interest rate 4\%, then we'd be better off reducing the amount of coal we sell today. If we wait for the future, then that would be a better investment than selling today. However, if everyone thinks this, then supply today falls, so that the price today rises, which means that the price today wasn't an equilibrium price, either. If the price today wasn't an equilibrium price, then the proposed price path can't be an equilibrium price path.\\
    
    When we have an extraction cost to incorporate, whenever we delay, we defer the profits from extracting but also the costs of extracting. The fact that deferring costs means they're less costly from a present-value standpoint makes us want to drag out the extraction and delay paying these costs. This condition ends up saying that, after adjusting for this cost delay, we end up indifferent between extracting and paying the cost now, or waiting to extract and also waiting to pay the costs, and this condition exactly balances the two.
    
    
\subsection{Net Present Value: Integrals and Sums:}\label{sec:npvintegrals}


    How did we know to take the sum for net present value in Equation~\eqref{eq:npvcoal}?
    
    This relates to your early days of calculus, back when integrals sounded like some kind of granola brand. When you took Riemann sums and calculated the area under a curve by making rectangles, and then taking the point on the curve that decided the height of your rectangle to be the left, the left, from the right, and the middle, you were using these rectangles to approximate the true integral.\\
    
    You may recall some tedious homework where you had to calculate this Riemann approximation for rectangles of length 1, length $\frac{1}{2}$, length $\frac{1}{4}$, and if your teacher was especially nasty even smaller.\\
    
    We're doing exactly the same type of approximation here, except our x-axis is in years, rather than units of distance, and our y-axis measures the net present value. The net present value curve is tracing out, at any instant $t$ in time, the value of the net present value for extracting that instantaneous resource.
    
    When you did this in early calc classes and counted these rectangles, in some cases you would overshoot, and in some you would undershoot. Do you remember when you would go over or under? Draw a picture of some curves and rectangle approximations if it's not coming to you. This will help you picture it visually. \\
    
    The summation notation, then, is just a bunch of rectangles approximating that integral curve. The integral is what we would get if we took the width of time and shrank it from 1 to infinitesimally small.
    
\end{subappendices}

\end{appendices}

\end{comment}



%% Bibliography

%\bibliography{bibliography/causal-inference.bib}



\end{document}


