\section{Creating Theorems, Hypotheses, and Lemmas}\label{theorems-hypotheses}

If you'd like to have something titled to which you can refer, like hypotheses 1-4, theorems, definitions, assumptions and so forth, you can create a command in the preamble that defines this new environment. 

For an intro tutorial, see \url{https://www.overleaf.com/learn/latex/theorems_and_proofs}, which has most of what you should need. Many examples below have been taken from here.

See \url{https://tex.stackexchange.com/questions/386905/hypothesis-with-amsthm-package}. You need \verb+amsthm+, and then check this preamble (the one in \verb+main-article+) in the area under the \verb+amsthm+ package. The high level of customization uses \verb+thmtools+, and the \verb+\cref+ uses the \verb+cleverref+ package, a package for fancy cross references. 

Note, though, that \verb+cleverref+ can clash with the \verb+cases+ package: \url{https://tex.stackexchange.com/questions/201437/cleveref-and-cases-packages-clash}. 

See \url{https://tex.stackexchange.com/questions/36295/cross-reference-packages-which-to-use-which-conflict} for tons of pros and cons you'll hopefully never need. 

The short: \verb+cleverref+ is a great package. Just make sure that \verb+cleverref+ is the last loaded cross-referencing package (the last one in your preamble)

Note also that the "hypothesis" environment used \verb+amsthm+, and the fancy cross referencing used \verb+cleverref+. You wouldn't need \verb+cleverref+ just to introduce the hypothesis environment, although it did require \verb+thmtools+ to make it so customized.

 \begin{hyp}[Test hypothesis] \label{hyp:a}This is my first hypothesis. \end{hyp}
 \begin{hyp} \label{hyp:b}This is my second hypothesis. \end{hyp}

  \Cref{hyp:a,hyp:b}.

Additionally, you can use the easier redefinition of commands from \verb+amsthm+ to designate theorems and lemmas that go under theorems, for instance:



\section{Examples}

\begin{theorem}[Test theorem]
This is the first theorem
\end{theorem}

\begin{corollary}
Here's a corollary to the theorem
\end{corollary}




\begin{theorem}
Let $f$ be a function whose derivative exists in every point, then $f$ is 
a continuous function.
\end{theorem}

\begin{theorem}[Pythagorean theorem]
\label{pythagorean}
This is a theorema about right triangles and can be summarised in the next 
equation 
\[ x^2 + y^2 = z^2 \]
\end{theorem}

And a consequence of theorem \ref{pythagorean} is the statement in the next 
corollary.

\begin{corollary}
There's no right rectangle whose sides measure 3cm, 4cm, and 6cm.
\end{corollary}

You can reference theorems such as \ref{pythagorean} when a label is assigned.

\begin{lemma}
Given two line segments whose lengths are $a$ and $b$ respectively there is a 
real number $r$ such that $b=ra$.
\end{lemma}

\begin{assumption}
I am making an assumption
\end{assumption}

\section{Restarting the examples}

\begin{theorem}
Let $f$ be a function whose derivative exists in every point, then $f$ is 
a continuous function.
\end{theorem}

You can reference theorems such as \ref{pythagorean} when a label is assigned.

\begin{lemma}
Given two line segments whose lengths are $a$ and $b$ respectively there is a 
real number $r$ such that $b=ra$.
\end{lemma}