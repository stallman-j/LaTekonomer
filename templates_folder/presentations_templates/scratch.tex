
\begin{frame}[label = lnmap]
\frametitle{Background: Location and Economy \hyperlink{data}{\beamerbutton{Back to Data}}}
\begin{figure}
\includegraphics[width=.6\textwidth]{figures/frog.jpg} \vspace*{-.5cm}\\%\hspace*{2cm} \\
{\tiny }
\end{figure}
{\tiny 
\begin{itemize}
\item - Mostly (over 90\%) agrarian
\item - Treaty of Tianjin opened Yingkou to trade in 1861
\end{itemize}
\hyperlink{outcomes}{\beamerbutton{Outcomes}}
\hyperlink{stylizedex}{\beamerbutton{Stylized Example}}}
\end{frame} 





\begin{comment}


\section{Framework}

\begin{frame}
\frametitle{How Do Households Allocate Resources?}
\onslide<1->{ Without social structure or norms, families would consider :
\begin{itemize}
\item Resource constraints
\item Gains to male v. female labor
\end{itemize}}
\onslide<2->{
Here, villages also have:
\begin{itemize}
\item Dowries: greater for richer families
\item Hypergamy: strategic sons v. daughters
\item Exogamous marriage: daughters leave the household
\end{itemize}}
\onslide<3->{
Norms according to ethnic / philosophical identification:
\begin{itemize}
\item Son preference {\tiny (Han yes, Manchu less)}
\item Hierarchy / filiality {\tiny (Han yes, Manchu less)}
\item Disapprove of women outside the home {\tiny (Han yes, Manchu less)}
\end{itemize}}

\end{frame}

\begin{comment}
\begin{frame}
\frametitle{Expectations:}
\onslide<1->{
\begin{itemize}
\item (1) Positive (negative) income shock:
\begin{itemize}
\item  - If just resource constraints and male labor: better (worse) relative mortality for women
\item  - Better relative outcomes in villages with more egalitarian preferences
\end{itemize}}
\onslide<2->{
\item (2) Changes in labor opportunities with commercialization
\begin{itemize}
\item  - Depends on relative gains to male v. female labor
\end{itemize}}
\onslide<3->{
\item (3) Better relative mortality for adult women in Han villages
\begin{itemize}
\item - Do not need to labor as hard
\item - Children more filial
\end{itemize}}

\end{itemize}
\onslide<4->{Testing requires both levels and interactions}
\end{frame}

\end{comment}



\begin{frame}
\frametitle{Economics and Culture in Regression Form}
\begin{block}{Full Specification Example}
\begin{equation}
\begin{split}
Y_{j,t} = \beta_0 + & \beta_1 trade\_shock_{j,t} + \beta_2 positions_{j,t} + \beta_3 grain\_prices_{j,t} 
\\ & + \beta_4 non\_Han_{j,t} + \beta_7 non\_Han_{j,t} \times trade\_shock
\\ & + \beta_8 non\_Han_{j,t} \times grain\_prices + \lambda_t + \alpha_j + \epsilon_{j,t}
\end{split}
\end{equation} 
\end{block}
{\tiny
\begin{itemize}
\item $Y_{j,t}$: mortality outcome in village $j$ in time $t$\\
{\tiny $Y>1$, relatively worse for women. $Y<1$, relatively better}
\item $trade\_shock_{j,t} $: trade shock proxy
\item $grain\_prices_{j,t}$ weather proxy
\item $positions_{j,t}$ : wealth proxy
\item $non\_Han_{j,t}$ : ethnicity proxy
\item $Confucian_{j,t}$ philosophy proxy 1
\item $Buddhist_{j,t}$ : philosophy proxy 2
\item $\lambda_t$ : time trend ; $\alpha_j$ village fixed effects; $\epsilon_{j,t}$ error term 
\end{itemize}}

\end{frame}

\section{Data}


\begin{frame}[label=data]
\frametitle{Data}
\begin{itemize} % full i
\item Outcomes: \hyperlink{mortratio}{\beamerbutton{Definition}} \hyperlink{stylizedex}{\beamerbutton{Stylized Example}}

\begin{itemize} % outcomes
\item China Multi-Generational Panel Dataset (CMGPD), Imperial Household Registers
\item Han Banner people, 1749 - 1909, triennial census
\item Mortality: \hyperlink{morttrends120}{\beamerbutton{Male-Female Trends}}\\
\end{itemize} % outcomes
\pause
\item Economic Variables:
\begin{itemize} %econ
\item Exports: Yingkou Imperial Maritime Customs \hyperlink{exports}{\beamerbutton{here}}
{\small
\begin{itemize} %trade 
\item Interact with distance \hyperlink{distance}{\beamerbutton{here}}
\item to proxy for trade shock \hyperlink{tradeshock}{\beamerbutton{here}}\\
\end{itemize}
} %trade
\pause

\item Monthly Prefectural Sorghum Prices from Qing Grain Price Records
{\small 
\begin{itemize} % sorghum
\item  Proxy for weather shocks\\
\end{itemize}} % sorghum

\pause
\item Proportion of population with a position in the Qing bureaucracy (CMGPD)
{\small 
\begin{itemize} %position
\item  Proxy for wealth
\end{itemize}} %position

\end{itemize} % econ
\end{itemize} % full

\end{frame}

\begin{frame}[label=data2]
\frametitle{Data con't}
\begin{itemize} % full
\item Cultural Variables:
\begin{itemize} % cultural vars
\item Non-Han naming (CMGPD): 
\begin{itemize} % non han
\item Proportion of population with a non-Han name
\item Naming trend \hyperlink{culture}{\beamerbutton{here}}
\end{itemize} % non han

\item Temple building (local gazetteers)
\begin{itemize}
\item No. of Confucian / Buddhist temples in 15 km
\item More \hyperlink{templesdata}{\beamerbutton{here}}
\end{itemize}
\end{itemize} % cultural vars
\end{itemize} % full

{\tiny Complications in the registers \hyperlink{ageofdeath}{\beamerbutton{here}}}

\end{frame}



\section{Results}


\begin{frame}[label = results16to20]
\frametitle{1749-1909, Outcome: Mortality Index, 16-20 \emph{sui} \hyperlink{datadefs}{\beamerbutton{Data Definitions}} \hyperlink{sumstats16to20}{\beamerbutton{Summary Stats}}}
\begin{table}\centering
\scalebox{0.5}{
 \begin{threeparttable}
 \vspace*{-1.7cm}\estauto{Tables/fm_nd_ratio_16_to_20full1.tex}{5}{S[table-format=1,table-column-width=2mm]}
\end{threeparttable}}
\end{table}
\parbox{\linewidth}{\tiny * p < 0.1, ** p < 0.05, *** p < 0.01. Standard errors in parentheses.\\
Age in \emph{sui} = Western age + 1.5 years
}
\end{frame}


\begin{frame}[label = results21to50]
\frametitle{1825-1906, Outcome: Mortality Index, 21-50 \emph{sui} \hyperlink{datadefs}{\beamerbutton{Data Definitions}} \hyperlink{sumstats21to50}{\beamerbutton{Summary Stats}}}
\begin{table}\centering
\scalebox{0.4}{
 \begin{threeparttable}
 \vspace*{-1.7cm}\estauto{Tables/fm_nd_ratio_21_to_50full2.tex}{5}{S[table-format=1,table-column-width=2mm]}
\end{threeparttable}}
\end{table}
\parbox{\linewidth}{\tiny * p < 0.1, ** p < 0.05, *** p < 0.01. Standard errors in parentheses.\\
Age in \emph{sui} = Western age + 1.5 years
}
\end{frame}



\begin{frame}
\frametitle{Results: Young Ages  \hyperlink{robustness}{\beamerbutton{Robustness Checks}}}
\begin{itemize}
\item Trade shock increase $\rightarrow$ 

\begin{itemize}
\item (1) Relative mortality worse for very young females
\item (2) Better for females in late teens
\end{itemize}
\pause
\item Possible channels:
{\small
\begin{itemize}
\item - Status differences for women working
\item - Dowries
\item - Hypergamy: sons as a status symbol
\item - More labor opportunities for females in late teens
\item - Teen daughters in households preferring daughters 
\end{itemize}
}

\end{itemize}
\end{frame}


\begin{frame}[label=resultsadults]
\frametitle{Results: Adult Mortality \hyperlink{robustness}{\beamerbutton{Robustness Checks}}}
\begin{itemize}
\item Grain price increase: relatively worse for women
\item Women in Han-like villages relatively better off than non-Han villages
\item ... but worse off in bad harvests
\pause
\item Possible channels:
{\small
\begin{itemize}
\item Male labor more valuable\\
\item Filiality to women (mothers) in Han-like villages\\
\item Han-like village preference for sons in bad years\\
\end{itemize}
}
\end{itemize}
\end{frame}



\section{Conclusion}

\begin{frame}
\frametitle{Conclusions}
\begin{itemize}
\item In certain age groups:
{\small
\begin{itemize}
\item - Trade shock and grain prices significant
\item - Temple construction not a strong measure
\item - Non-Han naming significant 
\end{itemize}
}
\pause
\item \textbf{Need to consider more than just infant mortality} %: can outcomes be better for women who do survive in socieities with excess female mortality?
\pause
\item Overall takeaways:
{\small
\begin{itemize}
\item - Incorporated both cultural and economic measures\\

\item - Tried to overcome data limitations \\

\item - Suggested new sources for quantiative analysis in Chinese economic history\\
\end{itemize}
}
\end{itemize}
\end{frame}




\begin{frame}
\frametitle{My thanks to:}

\begin{itemize}

\item Steve Nafziger\\
\bigskip
\item Jessica Leight and Sara LaLumia\\
\item The folks at the Oakley Center in 2015-2016
\item Many conversants: professors, colleagues, friends
\item Folks at and around Dabei Old Temple, Pujue Temple, Liaoning Provincial Archives, Liaoning Municipal Archives, Liaoning Municipal Library, Dalian Municipal Archives, Dalian Municipal Library
\item Professor Man He and parents
\item Professor Yu Li
\item Benefactors:
\begin{itemize}
\item The Barbara Solow Economic History Fellowship
\item The Ruchman Family
\end{itemize}
\pause
\item And y'all for making it out!
\end{itemize}


\end{frame}


\appendix
\section{Further Directions}
\begin{frame}
\frametitle{Further Directions}

\begin{itemize}
\item - Marriage Markets and Sex Ratios: temples had explanatory power\\
\item - Better outcome modeling (HLM)
\item - Further dynamics within older age brackets
\item - Comparison with other populations
\end{itemize}


\end{frame}

\section{Trends}

\begin{frame}[label = morttrends120]
\frametitle{Mortality for Males and Females \hyperlink{data}{\beamerbutton{Back to data}}}
\begin{figure}[h!]
\includegraphics[width=.9\textwidth]{mortratios1to20}\\
{\tiny }
\end{figure}
\end{frame}

\begin{frame}[label = morttrends2150]
\frametitle{Mortality for Males and Females}
\begin{figure}[h!]
\includegraphics[width=.9\textwidth]{mortratios21to50}\\
{\tiny }
\end{figure}
\hyperlink{data}{\beamerbutton{Go back to data}}
\end{frame}



\begin{frame}[label = exports]
\frametitle{Changes in exports \hyperlink{data}{\beamerbutton{Data}} \hyperlink{otherexportpics}{\beamerbutton{Other Export Graphs}}}
\begin{figure}[h!]
\includegraphics[width=.9\textwidth]{IMEXYKwantaels}\\
{\tiny }
\end{figure}

\end{frame}



\begin{frame}[label = distancelegend]
\frametitle{Trade Shock Values: Distance Legend \hyperlink{distance}{\beamerbutton{Map}}  \hyperlink{data}{\beamerbutton{Back to data}}} 
\begin{figure}[h!]
\includegraphics[width=.8\textwidth]{legenddistanceyk}\\
{\tiny }
\end{figure}
\end{frame}

\begin{frame}[label = distance]
\frametitle{Distance from Yingkou\\
\hyperlink{distancelegend}{\beamerbutton{Legend}} \hyperlink{data}{\beamerbutton{Back to data}}}
\begin{figure}[h!]
\includegraphics[width=.7\textwidth]{villagedistancetoykmap}\\
{\tiny }
\end{figure}
\end{frame}

\begin{frame}[label = tradeshocklegend]
\frametitle{Trade Shock: Legend \hyperlink{tradeshock}{\beamerbutton{Map}} \hyperlink{data}{\beamerbutton{Back to data}}}
\begin{figure}[h!]
\includegraphics[width=.8\textwidth]{tradeshocklegend}\\
{\tiny }
\end{figure}
\end{frame}

\begin{frame}[label = tradeshock]
\frametitle{The Trade Shock: \hyperlink{tradeshocklegend}{\beamerbutton{Legend}} \hyperlink{data}{\beamerbutton{Back to data}}}
\begin{figure}[h!]
\includegraphics[width=.7\textwidth]{tradeshockallyears}\\
\end{figure}
\end{frame}




\begin{frame}[label=otherexportpics]
\frametitle{Export Trends: Other Graphs}
\begin{figure}[h!]
\includegraphics[width=.85\textwidth]{IMEXYKlntaels}\\
\end{figure}
{\small 
Back to \hyperlink{data}{\beamerbutton{data}}.}
\end{frame}

\begin{frame}[label=otherexportpics]
\frametitle{Export Trends: Other Graphs}
\begin{figure}[h!]
\includegraphics[width=.85\textwidth]{SoybeanExportsfromYingkou}\\
\end{figure}
{\small 
Back to \hyperlink{data}{\beamerbutton{data}}.}
\end{frame}

\begin{frame}[label = culture]
\frametitle{Changes in Naming Practices \hyperlink{data2}{\beamerbutton{Back to data}}}
\begin{figure}[h!]
\includegraphics[width=.85\textwidth]{nonhanproptime}\\
{\tiny Proportion: number of males with a non-Han name relative to the population}

\end{figure}

\end{frame}



\section{Temples}

\begin{frame}[label = temples]
\frametitle{Creating the Temples Data (1)  \hyperlink{data2}{\beamerbutton{Back to data}}}
Start with a bunch of these:
\begin{figure}[h!]
\includegraphics[width=.7\textwidth]{dabeigusi2}\\
{\tiny Source: Baidu.com}

\end{figure}

\end{frame}

\begin{frame}[label = temples2]
\frametitle{Creating the Temples Data (2)  \hyperlink{data2}{\beamerbutton{Back to data}}}
They got recorded like so:
\begin{figure}[h!]
\includegraphics[width=.7\textwidth]{SYtempleex}\\
{\tiny Source: \emph{Fengtian Tongzhi}}

\end{figure}

\end{frame}

\begin{frame}[label = temples3]
\frametitle{Creating the Temples Data (3)  \hyperlink{data2}{\beamerbutton{Back to data}}}
Find and geolocate from here:
\begin{figure}[h!]
\includegraphics[width=.6\textwidth]{FTTZcountymap2}\\
{\tiny Source: \emph{Fengtian Tongzhi}}
\end{figure}
And Baidu, of course.

\end{frame}



\begin{frame}[label=confucianlegend]
\frametitle{Confucian Temples \hyperlink{templeculture}{\beamerbutton{Maps!}} \hyperlink{data2}{\beamerbutton{Back to data}}}
\begin{figure}[h!]
\includegraphics[width=.7\linewidth]{confuciantemplelegend}\\
\end{figure}
\end{frame}





\begin{frame}[label = templeculture]
\frametitle{Located Confucian Temples before 1750, 1800, 1909 \hyperlink{confucianlegend}{\beamerbutton{Legend}}  \hyperlink{data2}{\beamerbutton{Back to data}}}
\begin{figure}
    \begin{center}
\includegraphics<1>[width=.6\linewidth]{confuciantemplepre1750}
\includegraphics<2>[width=.6\linewidth]{confuciantemplepre1800}
\includegraphics<3>[width=.6\linewidth]{confuciantempleallyears}
    \end{center}
\end{figure}
\end{frame}



\begin{frame}[label=buddhistlegend]
\frametitle{Buddhist Temples  \hyperlink{buddhistculture}{\beamerbutton{Maps!}}\hyperlink{data2}{\beamerbutton{Back to data}}}
\begin{figure}[h!]
\includegraphics[width=.7\linewidth]{buddhismlegendreal}\\
{\tiny }

\end{figure}

\end{frame}


\begin{frame}[label = buddhistculture]
\frametitle{Located Buddhist Temples before 1750, 1800, 1909 \hyperlink{buddhistlegend}{\beamerbutton{Legend}}  \hyperlink{data2}{\beamerbutton{Back to data}}}
\begin{figure}
    \begin{center}
\includegraphics<1>[width=.6\linewidth]{buddhismpre1750}
\includegraphics<2>[width=.6\linewidth]{buddhismpre1800}
\includegraphics<3>[width=.6\linewidth]{buddhismallyears}
    \end{center}
\end{figure}
\end{frame}


\begin{frame}[label=imrlegend]
\frametitle{Infant Mortality Ratio Legend  \hyperlink{imrratio}{\beamerbutton{Back to the maps}}}
\begin{figure}[h!]
\includegraphics[width=\linewidth]{imrlegendfinal2}\\
{\tiny }

\end{figure}

\end{frame}



\begin{frame}[label=2013lifeexpectancylegend]
\frametitle{Infant Mortality Ratio Legend  \hyperlink{2013lifeexpectancy}{\beamerbutton{Back to the map}}}
\begin{figure}[h!]
\includegraphics[width=.5\linewidth]{2013lifeexpectancylegend}\\
{\tiny }

\end{figure}

\end{frame}


\section{Data Issues}

\begin{frame}[label = ageofdeath]
\frametitle{Issues with Accounting for Mortality \hyperlink{data2}{\beamerbutton{Back to data}}}
\begin{figure}[h!]
\includegraphics[width=.7\textwidth]{DeathAgeFinal}\\
{\tiny }

\end{figure}

\end{frame}

\begin{frame}[label = obsbyyear]
\frametitle{Issues with Observations \hyperlink{data2}{\beamerbutton{Back to data}}}
Number of Individual Observations by Year
\begin{figure}[h!]
\includegraphics[width=.7\textwidth]{obsbyyear}\\
{\tiny Data Source: CMGPD 2011}

\end{figure}

\end{frame}

\begin{frame}[label = obsbyyear]
\frametitle{Issues with Registers \hyperlink{data2}{\beamerbutton{Back to data}}}
Number of Registers Recorded by Year
\begin{figure}[h!]
\includegraphics[width=.7\textwidth]{registersbyyear}\\
{\tiny Data Source: CMGPD 2011}

\end{figure}

\end{frame}



\section{Outcomes}
\begin{frame}[label=mortratio]
\frametitle{Outcomes: Female-Male Mortality Index \hyperlink{data}{\beamerbutton{Back to Data}} }
\begin{block}{Mortality by sex}
\begin{equation}
Deaths \ ratio = \frac{deaths_{t+3,sex}}{present_{t,sex}}
\end{equation}
\end{block}

Where:\\
\begin{itemize}
\item $sex$ denotes male or female\\
\item $deaths$ is the number of deaths of a given $sex$ between time $t$ and $t+3$\\
\item $present$: observe in time $t$ and time $t+3$
\end{itemize}

\end{frame}



\section{Summary Stats}





\begin{frame}[label = sumstats16to20]
\frametitle{Summary Stats for Mortality Index, 16-20 \emph{sui} \hyperlink{datadefs}{\beamerbutton{Data Definitions}} \hyperlink{results16to20}{\beamerbutton{16-20 sui results}}}
{\tiny \input{Tables/sumstats1620nograin.tex}}
\end{frame}

\begin{frame}[label = sumstats21to50]
\frametitle{Summary Stats for Mortality Index, 21-50 \emph{sui} \hyperlink{datadefs}{\beamerbutton{Data Definitions}} \hyperlink{results21to50}{\beamerbutton{21-50 sui results}}}
{\tiny \input{Tables/sumstats2150withgrain.tex}}
\end{frame}

\begin{frame}[label = sumstats]
\frametitle{Summary Stats \hyperlink{datadefs}{\beamerbutton{Data Definitions}} \hyperlink{results}{\beamerbutton{Results}}}
{\tiny \input{Tables/presentationsum.tex}}
\end{frame}

\begin{frame}[label=robustness]
\frametitle{Robustness Checks  \hyperlink{resultsadults}{\beamerbutton{Back to Results}}}
Robust to:
\begin{itemize}
\item - Different definitions of age brackets (1-17, 18-46, 47-80)
\item - Taking out non-Han-named people with positions (omitted income variable)
\item - Removing married daughters in young age brackets
\item - Different definitions of non-Han variables
\item - Removing extreme villages / observations of mortality index
\item - Remove after 1888
\item - Interpolate v. assign grain prices of the nearest prefectural seat
\end{itemize}
\end{frame}

\end{comment}